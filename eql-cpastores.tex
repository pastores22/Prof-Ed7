% Options for packages loaded elsewhere
\PassOptionsToPackage{unicode}{hyperref}
\PassOptionsToPackage{hyphens}{url}
\PassOptionsToPackage{dvipsnames,svgnames,x11names}{xcolor}
%
\documentclass[
  letterpaper,
  DIV=11,
  numbers=noendperiod]{scrartcl}

\usepackage{amsmath,amssymb}
\usepackage{iftex}
\ifPDFTeX
  \usepackage[T1]{fontenc}
  \usepackage[utf8]{inputenc}
  \usepackage{textcomp} % provide euro and other symbols
\else % if luatex or xetex
  \usepackage{unicode-math}
  \defaultfontfeatures{Scale=MatchLowercase}
  \defaultfontfeatures[\rmfamily]{Ligatures=TeX,Scale=1}
\fi
\usepackage{lmodern}
\ifPDFTeX\else  
    % xetex/luatex font selection
\fi
% Use upquote if available, for straight quotes in verbatim environments
\IfFileExists{upquote.sty}{\usepackage{upquote}}{}
\IfFileExists{microtype.sty}{% use microtype if available
  \usepackage[]{microtype}
  \UseMicrotypeSet[protrusion]{basicmath} % disable protrusion for tt fonts
}{}
\makeatletter
\@ifundefined{KOMAClassName}{% if non-KOMA class
  \IfFileExists{parskip.sty}{%
    \usepackage{parskip}
  }{% else
    \setlength{\parindent}{0pt}
    \setlength{\parskip}{6pt plus 2pt minus 1pt}}
}{% if KOMA class
  \KOMAoptions{parskip=half}}
\makeatother
\usepackage{xcolor}
\setlength{\emergencystretch}{3em} % prevent overfull lines
\setcounter{secnumdepth}{-\maxdimen} % remove section numbering
% Make \paragraph and \subparagraph free-standing
\ifx\paragraph\undefined\else
  \let\oldparagraph\paragraph
  \renewcommand{\paragraph}[1]{\oldparagraph{#1}\mbox{}}
\fi
\ifx\subparagraph\undefined\else
  \let\oldsubparagraph\subparagraph
  \renewcommand{\subparagraph}[1]{\oldsubparagraph{#1}\mbox{}}
\fi


\providecommand{\tightlist}{%
  \setlength{\itemsep}{0pt}\setlength{\parskip}{0pt}}\usepackage{longtable,booktabs,array}
\usepackage{calc} % for calculating minipage widths
% Correct order of tables after \paragraph or \subparagraph
\usepackage{etoolbox}
\makeatletter
\patchcmd\longtable{\par}{\if@noskipsec\mbox{}\fi\par}{}{}
\makeatother
% Allow footnotes in longtable head/foot
\IfFileExists{footnotehyper.sty}{\usepackage{footnotehyper}}{\usepackage{footnote}}
\makesavenoteenv{longtable}
\usepackage{graphicx}
\makeatletter
\def\maxwidth{\ifdim\Gin@nat@width>\linewidth\linewidth\else\Gin@nat@width\fi}
\def\maxheight{\ifdim\Gin@nat@height>\textheight\textheight\else\Gin@nat@height\fi}
\makeatother
% Scale images if necessary, so that they will not overflow the page
% margins by default, and it is still possible to overwrite the defaults
% using explicit options in \includegraphics[width, height, ...]{}
\setkeys{Gin}{width=\maxwidth,height=\maxheight,keepaspectratio}
% Set default figure placement to htbp
\makeatletter
\def\fps@figure{htbp}
\makeatother

\KOMAoption{captions}{tableheading}
\makeatletter
\makeatother
\makeatletter
\makeatother
\makeatletter
\@ifpackageloaded{caption}{}{\usepackage{caption}}
\AtBeginDocument{%
\ifdefined\contentsname
  \renewcommand*\contentsname{Table of contents}
\else
  \newcommand\contentsname{Table of contents}
\fi
\ifdefined\listfigurename
  \renewcommand*\listfigurename{List of Figures}
\else
  \newcommand\listfigurename{List of Figures}
\fi
\ifdefined\listtablename
  \renewcommand*\listtablename{List of Tables}
\else
  \newcommand\listtablename{List of Tables}
\fi
\ifdefined\figurename
  \renewcommand*\figurename{Figure}
\else
  \newcommand\figurename{Figure}
\fi
\ifdefined\tablename
  \renewcommand*\tablename{Table}
\else
  \newcommand\tablename{Table}
\fi
}
\@ifpackageloaded{float}{}{\usepackage{float}}
\floatstyle{ruled}
\@ifundefined{c@chapter}{\newfloat{codelisting}{h}{lop}}{\newfloat{codelisting}{h}{lop}[chapter]}
\floatname{codelisting}{Listing}
\newcommand*\listoflistings{\listof{codelisting}{List of Listings}}
\makeatother
\makeatletter
\@ifpackageloaded{caption}{}{\usepackage{caption}}
\@ifpackageloaded{subcaption}{}{\usepackage{subcaption}}
\makeatother
\makeatletter
\@ifpackageloaded{tcolorbox}{}{\usepackage[skins,breakable]{tcolorbox}}
\makeatother
\makeatletter
\@ifundefined{shadecolor}{\definecolor{shadecolor}{rgb}{.97, .97, .97}}
\makeatother
\makeatletter
\makeatother
\makeatletter
\makeatother
\ifLuaTeX
  \usepackage{selnolig}  % disable illegal ligatures
\fi
\IfFileExists{bookmark.sty}{\usepackage{bookmark}}{\usepackage{hyperref}}
\IfFileExists{xurl.sty}{\usepackage{xurl}}{} % add URL line breaks if available
\urlstyle{same} % disable monospaced font for URLs
\hypersetup{
  pdftitle={exploring Quarto and Latex},
  pdfauthor={Carlo Pastores Jr.},
  colorlinks=true,
  linkcolor={blue},
  filecolor={Maroon},
  citecolor={Blue},
  urlcolor={Blue},
  pdfcreator={LaTeX via pandoc}}

\title{exploring Quarto and Latex}
\author{Carlo Pastores Jr.}
\date{}

\begin{document}
\maketitle
\ifdefined\Shaded\renewenvironment{Shaded}{\begin{tcolorbox}[boxrule=0pt, frame hidden, borderline west={3pt}{0pt}{shadecolor}, interior hidden, enhanced, breakable, sharp corners]}{\end{tcolorbox}}\fi

\hypertarget{antidifferentiation-and-indefinite-integrals}{%
\section{4.1 ANTIDIFFERENTIATION AND INDEFINITE
INTEGRALS}\label{antidifferentiation-and-indefinite-integrals}}

\hypertarget{integration-by-substitution}{%
\section{Integration by
Substitution}\label{integration-by-substitution}}

\hypertarget{htm:subrule}{}
\hypertarget{substitution-rule}{%
\subsection{4.1.2(Substitution Rule)}\label{substitution-rule}}

If \(u=g(x)\) is a differentiable function whose range is an interval
\(I\) and \(f\) is a continuous on \(I\) then,

\[
\int f(g(x))\cdot g'(x)dx = \int f(u)du
\] \textbf{Example 4.1.12.}

1.\(\int \left(1-4x\right)^\frac{1}{2}dx\)

If we let \(u=1-4x\), then \(du=-4dx\). We multiply the integrand
\(\frac{-4}{-4}\). Thus, \[
\int \left(1-4x\right)^\frac{1}{2}dx = \int \left(1-4x\right)^\frac{1}{2}\cdot \frac{-4}{-4}dx = \int u^\frac{1}{2} \left( -\frac{du}{4} \right)= -\frac{1}{4} \int u^\frac{1}{2} du = -\frac{1}{4}\cdot \frac{2u^\frac{3}{2}}{3} + C. \]

We put the final answer in terms of \(x\) by substituting \(u=1-4x\).
Therefore, \[
\int \left(1-4x\right)^\frac{1}{2} dx = \frac{\left(1-4x\right)^\frac{3}{2}}{6} + C.\]

2.\(\int x^2\left(x^3-1\right)^{10} dx\)

Let \(u=x^3-1\). Then \(du=3x^2dx\), or \(\frac{du}{3}=x^2 dx\). By
substitution, \[
\int x^2\left(x^3-1\right)^{10} dx = \int u^{10}\cdot \frac{du}{3}= \frac{1}{3}\int u^{10}du=\frac{u^{11}}{33} + C= \frac{\left(x^3-1\right)^{11}}{33}+ C.
\]

3.\(\int \frac{x}{\left(x^2+1\right)^3} dx\)

Let \(u=x^2+1\). Then \(du=2xdx\), or \(\frac{du}{2}=xdx\). By
substitution, \[
\int \frac{x}{\left(x^2+1\right)^3}dx=\frac{1}{2}\int u^{-3}du=\frac{1}{2}\cdot \frac{u^{-2}}{-2}+C=-\frac{1}{4\left(x^2+1\right)^2}+C.
\]

4.\(\int \cos^4 x \sin xdx\)

Let \(u=\cos x\). Then \(du=-\sin x dx\), or \(-du=\sin xdx\). By
substitution, \[
\int \cos^4 x \sin xdx=-\int u^4 du=-\frac{u^5}{5}+C=-\frac{\cos^5 x}{5}+C.
\]

5.\(\int x \sec^3 \left( x^2 \right)\tan\left(x^2\right)dx\)

Let \(u=\sec\left (x^2\right)\). Then
\(du=\sec\left(x^2\right)\tan\left(x^2\right)\cdot 2xdx\), or
\(\frac{du}{2}=\sec\left (x^2\right)\tan\left(x^2\right)\cdot xdx\). By
substitution,

\[
\begin{aligned}
\int x\sec^3 \left(x^2\right)\tan\left(x^2\right)dx &= \int \sec^2 \left(x^2\right)\sec\left(x^2\right)\tan\left(x^2\right)\cdot xdx \\ &= \int u^2du=\frac{1}{2}\cdot \frac{u^3}{3}+C \\& = \frac{\sec^3 \left(x^2\right)}{6} + C.
\end{aligned}
\]

6.\(\int \frac{\tan\frac{1}{s}+\tan\frac{1}{s}\sin\frac{1}{s}}{s^2\cos\frac{1}{s}}ds\)
Let \(u=\frac{1}{s}\). Then \(du=-\frac{1}{s^2}ds\) or
\(-du=\frac{ds}{s^2}\). By substitution, \[
\begin{aligned}
\int \frac{\tan\frac{1}{s}+\tan\frac{1}{s}\sin\frac{1}{s}}{s^2\cos\frac{1}{s}}ds &= -\int \frac{\tan u+\tan u \sin u}{\cos u}du\\&=-\int (\sec u \tan u + \tan^2 u)du\\&=-\int (\sec u \tan u +\sec^2 u -1 )du\\&=-(\sec u +\tan u-u) + C\\&=-\sec\frac{1}{s}-\tan\frac{1}{s}+\frac{1}{s}+C.  
\end{aligned}
\]

7.\(\int t\sqrt {t-1} dt\)

Let \(u=t-1\). Then \(u=dt\). Also, \(t=u+1\). By substitution,
\[ \begin{aligned}
\int t\sqrt{t-1} dt &=\int \left(u+1\right)u^\frac{1}{2}du =\int \left( u^\frac{3}{2}+u^\frac{1}{2} \right)du =\frac{2u^\frac{5}{2}}{5}+\frac{2u^\frac{3}{2}}{3}+C\\&=\frac{2\left(t-1\right)^\frac{5}{2}}{5}+\frac{2\left(t-1\right)^\frac{3}{2}}{3}+C.
\end{aligned}
\]

8.\(\int \frac{t^3}{\sqrt{t^2+3}}dt\)

Let \(u=t^2+3\). Then \(du=2tdt\), or \(\frac{du}{2}=tdt\). Also,
\(t^2=u-3\). By substitution, \[
\begin{aligned}
\int \frac{t^3}{\sqrt{t^2+3}}dt &= \int \frac{t^2\cdot{t}}{\sqrt {t^2+3}}dt=\int u^\frac{-1}{2}(u-3)\frac{du}{2}&\\&=\frac{1}{2}\int \left(u^\frac{1}{2}-3u^\frac{-1}{2}\right)du =\frac{1}{2} \left(\frac{2u^\frac{3}{2}}{3}-6u^\frac{1}{2}\right)+C \\& =\frac{\left(t^2+3\right)^\frac{3}{2}}{3}-3\left(t^2+3\right)^\frac{1}{2}+C.
\end{aligned}
\]

9.\(\int \sqrt{4+\sqrt x}dx\)

Let \(u=4+\sqrt x\). Then \(du=\frac{1}{2\sqrt x}dx\) or
\(2du=\frac{dx}{\sqrt x}\). By substitution, \[
\begin{aligned}
\int \sqrt{4+\sqrt x}dx&=\int \sqrt{4+\sqrt x} \cdot\frac{\sqrt x}{\sqrt x}dx\\&=\int \sqrt{4+\sqrt x} \cdot \sqrt x\cdot \frac{dx}{\sqrt x} (\sqrt x=u-4)\\&=\int u^\frac{1}{2}\cdot (u-4)\cdot 2du\\&=\int(2u^\frac{3}{2}-8u^\frac{1}{2}) du\\&=\frac{2\cdot2u^\frac{5}{2}}5-\frac{2\cdot 8u^\frac{3}{2}}{3}+C\\&=\frac{4\left (4+\sqrt x\right)^\frac{5}{2}}{5}-\frac{16\left(4+\sqrt x\right)^\frac{3}{2}}{3}+C.
\end{aligned}
\]

\hypertarget{htm:ParticularAntiderivatives}{}
\hypertarget{particular-antiderivatives}{%
\subsubsection{Particular
Antiderivatives}\label{particular-antiderivatives}}

Now suppose that given a function \(f(x)\), we wish to find a particular
antiderivative \(F(x)\) of \(f(x)\) that satisfies a given condition.
Such a condition is called an initial or boundary condition.

\hypertarget{exm:particular_derivatives}{}
\begin{enumerate}
\def\labelenumi{\arabic{enumi}.}
\item
  Given that \(F'(x)=2x\) and \(F(2)=6\), find \(F(x)\).

  Solution.

  Since \(F'(x)=2x\), we have \[
  F(x)=\int 2xdx=x^2+C.
  \]The initial condition \(F(2)=6\) implies that \(F(2)=2^2+C=6\). We
  get \(C=2\). Therefore, the particular antiderivative that we wish to
  find is \[
  F(x)=x^2+2.
  \]
\item
  The slope of the the tangent line at any point \((x,y)\) on a curve is
  given by \(3\sqrt x\). Find an equation of the curve if the point
  \((9,4)\) is on the curve.

  Solution.

  Let \(y=F(x)\) be an equation of the curve. The slope of the tangent
  line \(m_{TL}\) at a point \((x,y)\) on the graph of the curve is
  given by \(F'(x)=3\sqrt x\). We have \[
  F(x)=\int 3x^\frac{1}{2}dx=2x^\frac{3}{2}+C.
  \] The initial condition that \((9,4)\) is on the curve implies that
  \(F(9)=2\cdot 9^\frac{3}{2}=4\). We obtain \(C=-50\). Thus, an
  equation of the curve is \[
  y=2x^\frac{3}{2}=-50.
  \]
\end{enumerate}

\hypertarget{indeterminate-forms-and-lhospitals-rule}{%
\subsubsection{Indeterminate Forms and L'Hospital's
Rule}\label{indeterminate-forms-and-lhospitals-rule}}

\hypertarget{htm:Indeterminateforms}{}
\hypertarget{indeterminate-forms-of-type}{%
\subsubsection{Indeterminate Forms of
Type}\label{indeterminate-forms-of-type}}

\(\frac{0}{0}\) and \(\frac{\infty}{\infty}\)

We began this course with the concept of the limit: the behavior of a
function as the independent variable approaches a certain value, or as
it increases or decreases without bound. We saw tangent lines, rates of
change, and areas of plane regions, as limits of certain quantities.
Indeed, the concept of the limit is the central idea about which the
entire calculus revolves.

Now, we conclude this course by revising this fundamental idea. We shall
see that, with the aid of derivatives, certain limits can be evaluated
more conveniently. Before proceeding, we first recall some terminology
defined in the early part of this course. We also recall here some
techniques in evaluating limits we have previously encountered.

\hypertarget{def:Definition}{}
\hypertarget{definition}{%
\subsubsection{Definition}\label{definition}}

The \(\displaystyle\lim_{x \to a} \frac{f(x)}{g(x)}\) is an
indeterminate form of type 1. \(\frac{0}{0}\) if
\(\displaystyle\lim_{x \to a} f(x) = \displaystyle\lim_{x \to a} g(x)=0\).
2. \(\frac{\infty}{\infty}\) if \(\displaystyle\lim_{x \to a} f(x)\) and
\(\displaystyle\lim_{x \to a} g(x)\) are both \(+\infty\) or
\(-\infty\). Of course, \("x \to a"\) may be replaced by
\("x \to a^+"\), \("x \to a^-"\), \("x \to +\infty"\) or
\("x \to -\infty"\).

\leavevmode\vadjust pre{\hypertarget{exm:definition}{}}%
Evaluate the following limits.

\begin{enumerate}
\def\labelenumi{\arabic{enumi}.}
\item
  \(\displaystyle\lim_{x \to 0} \frac{x^2-3x}{2x^2+x}\)
  \(\left(\frac{0}{0}\right)\) Solution. \[
  \displaystyle\lim_{x \to 0} \frac{x^2-3x}{2x^2+x}= \displaystyle\lim_{x \to 0} \frac{x(x-3)}{x(2x+1)}=\displaystyle\lim_{x \to 0} \frac{x-3}{2x+1}=-3
  \]
\item
  \(\displaystyle\lim_{x \to 0} \frac{\sin5x}{\sin3x}\)
  \(\left(\frac{0}{0}\right)\) Solution. \[
  \displaystyle\lim_{x \to 0} \frac{\sin5x}{\sin3x}=\displaystyle\lim_{x \to 0} \left(\frac{\sin5x}{5x}\right)\left(\frac{3x}{\sin3x}\right)\left(\frac{5}{3}\right)=1\cdot1\cdot\frac{5}{3}=\frac{5}{3}
  \]
\item
  \(\displaystyle\lim_{x \to 2} \frac{x^2+3x-10}{x^2-4x+4}\)
  \(\left(\frac{0}{0}\right)\) Solution. \[
  \begin{aligned}
  \displaystyle\lim_{x \to 2} \frac{x^2+3x-10}{x^2-4x+4}&=\displaystyle\lim_{x \to 2}\frac{(x+5)(x-2)}{(x-2)^2}\\&=\displaystyle\lim_{x \to 2}\frac{x+5}{x-2}\\&=-\infty
  \end{aligned}
  \]
\item
  \(\displaystyle\lim_{x \to +\infty}\frac{3x-1}{7-6x}\)
  \(\left(\frac{+\infty}{-\infty}\right)\) Solution. \[
  \displaystyle\lim_{x \to +\infty}\frac{3x-1}{7-6x}=\displaystyle\lim_{x \to +\infty}\frac{3x-1}{7-6x}\cdot \frac{\frac{1}{x}}{\frac{1}{x}}=\displaystyle\lim_{x \to +\infty}\frac{3-\frac{1}{x}}{\frac{7}{x}-6}=-\frac{1}{2}
  \]
\end{enumerate}

\hypertarget{htm:lhospitalsrule}{}
\hypertarget{lhospitals-rule}{%
\subsubsection{L'Hospital's Rule}\label{lhospitals-rule}}

The following theorem tells us how derivatives can be used to evaluate
limits that are indeterminate of type \(\frac{0}{0}\) or
\(\frac{\infty}{\infty}\). It usually referred to as L'Hospital's Rule,
after the French mathematician Guillaume Francois Marquis de L'Hospital.

\leavevmode\vadjust pre{\hypertarget{htm:theorem}{}}%
Let \(f\) and \(g\) be functions differentiable on an open interval
\(I\) containing \(a\) except possibly at \(a\) and \(g'(x)\ne 0\) for
all \(x \in I \setminus {a}\). If
\(\displaystyle\lim_{x \to a}\frac{f(x)}{g(x)}\) is indeterminate of
type \(\frac{0}{0}\) or \(\frac{\infty}{\infty}\), then \[
\displaystyle\lim_{x \to a}\frac{f(x)}{g(x)}=\displaystyle\lim_{x \to a}\frac{f'(x)}{g'(x)}
\] provided \(\displaystyle\lim_{x \to a}\frac{f'(x)}{g'(x)}\) exists or
\(\displaystyle\lim_{x \to a}\frac{f'(x)}{g'(x)}=\pm \infty\).

\leavevmode\vadjust pre{\hypertarget{htm:remarks}{}}%
Remarks L'Hospital's Rule, with suitable modifications, is valid if
\("x \to a"\) is replaced by \("x \to a^+"\), \("x \to a^-"\),
\("x \to +\infty"\) or \("x \to -\infty"\).

\leavevmode\vadjust pre{\hypertarget{exm:remarks}{}}%
Example Evaluate the following limits. 1.
\(\displaystyle\lim_{x \to 0} \frac{x^2-3x}{2x^2+x}\) \(\frac{0}{0}\)
Solution. \[
\displaystyle\lim_{x \to 0} \frac{x^2-3x}{2x^2+x}=\displaystyle\lim_{x \to 0} \frac{D_x\left(x^2-3x\right)}{D_y\left(2x^2+x\right)}=\displaystyle\lim_{x \to 0} \frac{2x-3}{4x+1}=-3
\]

\begin{enumerate}
\def\labelenumi{\arabic{enumi}.}
\setcounter{enumi}{1}
\item
  \(\displaystyle\lim_{x \to 0} \frac{\sin5x}{\sin3x}\)
  \(\left(\frac{0}{0}\right)\) Solution. \[
  \displaystyle\lim_{x \to 0} \frac{\sin5x}{\sin3x}=\displaystyle\lim_{x \to 0} \frac{5\cos5x}{3\cos3x}=\frac{5}{3}
  \]
\item
  \(\displaystyle\lim_{x \to 2^-} \frac{x^2+3x-10}{x^2-4x+4}\)
  \(\left(\frac{0}{0}\right)\) Solution. \[
  \begin{aligned}
  \displaystyle\lim_{x \to 2^-} \frac{x^2+3x-10}{x^2-4x+4}&=\displaystyle\lim_{x \to 2^-}\frac{2x+3}{2x-4}\\&\\&=-\infty
  \end{aligned}
  \]
\item
  \(\displaystyle\lim_{x \to +\infty}\frac{3x-1}{7-6x}\)
  \(\left(\frac{+\infty}{-\infty}\right)\) Solution. \[
  \displaystyle\lim_{x \to +\infty}\frac{3x-1}{7-6x}=\displaystyle\lim_{x \to +\infty} \frac{3}{-6}=-\frac{1}{2}
  \]
\item
  \(\displaystyle\lim_{x \to 1} \frac{x^3-3x+2}{1-x+\ln x}\)
  \(\left(\frac{0}{0}\right)\) Solution. \[
  \begin{aligned}
  \displaystyle\lim_{x \to 1} \frac{x^3-3x+2}{1-x+\ln x}&=\displaystyle\lim_{x \to 1}\frac{3x^2-3}{-1+\frac{1}{x}}\\&=\displaystyle\lim_{x \to 1}\frac{6x}{-\frac{1}{x^2}}\\&=-6
  \end{aligned}
  \]
\item
  \(\displaystyle\lim_{x \to 0^-}\frac{\csc x}{1-\cot x}\)
  \(\left(\frac{-\infty}{+\infty}\right)\) Solution. \[
  \begin{aligned}
  \displaystyle\lim_{x \to 0^-}\frac{\csc x}{1-\cot x}&=\displaystyle\lim_{x \to 0^-}\frac{-\csc x \cot x}{\csc^2 x}\\&=\displaystyle\lim_{x \to 0^-}\frac{-\cot x}{\csc x}\\&=\displaystyle\lim_{x \to 0^-} \frac{\csc^2 x}{-\csc x \cot x}\\&=\displaystyle\lim_{x \to 0^-}\frac{\csc x}{-\cot x}\\&=\displaystyle\lim_{x \to 0^-}\frac{-\cot x}{\csc x}
  \end{aligned}
  \]
\end{enumerate}

Observe that the expression in the last line above is exactly the same
as the expression in the second line. Hence, continued application of
L'Hospital's Rule here will just lead to an infinite string of equations
and will not help us evaluate the limit. This example should make you
realize that L'Hospital's Rule is not always helpful. Sometimes, we just
have to use some old-fashioned tricks. For instance, we evaluate this
limit by simply manipulating the given expression to obtain a simpler
expression: \[
\begin{aligned}
\displaystyle\lim_{x \to 0^-}\frac{\csc x}{1-\cot x}&=\displaystyle\lim_{x \to 0^-}\frac{\frac{1}{\sin x}}{1-\frac{\cos x}{\sin x}}\\&=\displaystyle\lim_{x \to 0^-}\frac{1}{\sin x-\cos x}\\&=-1
\end{aligned}
\]

It is imperative to remember the behavior of each function introduced in
this course; doing so will help us in computing new limits. Recalling
the graphs of our new functions will be helpful in remembering their
behavior. For instance, using the graph of \(f(x)=\log_a x\), where
\(0\lt a \lt 1\), one sees that
\(\displaystyle\lim_{x \to 0^+}\log_a x=+\infty\). Thus if \(f(x)\)
approaches \(0\) through positive values \(x\) approaches \(k\), then \[
\displaystyle\lim_{x \to k}\log_a[f(x)]=\displaystyle\lim_{y \to 0^+}\log_a y=+\infty.
\]

\leavevmode\vadjust pre{\hypertarget{exm:5.6.6}{}}%
Solution Note that since \(x+\sin x \ge x - 1\) for any \(x \in R\) and
\(\displaystyle\lim_{x \to +\infty} {x-1=+\infty}\), then
\(\displaystyle\lim_{x \to +\infty}{x + \sin x = +\infty}\) as well.
Thus, the limit is indeterminate of type
\(\left(\frac{\infty}{\infty}\right)\).

However,
\(\displaystyle\lim_{x \to +\infty}{\frac{D_x(x+\sin x)}{D_x(x)}=\displaystyle\lim_{x \to +\infty}\frac{1+\cos x}{1}}\)
does not exist \(\cos x\) does not approach any particular value as
\(x\to +\infty\). Neither does \(1+\cos x\) grow without bound. Thus,
L'Hospital's Rule does not apply.

We therefore need to employ other techniques. In particular, notice that
any \(x\gt 0\), the following hold: \[
\begin{aligned}
x-1 \le x+\sin x \le x+1 \\
x-1 \le x+\sin x \le x+1\\
\iff x
\end{aligned}
\]

Also,
\(\displaystyle\lim_{x\to+\infty}{\frac{x-1}{x}=1=\displaystyle\lim_{x\to+\infty}{\frac{x+1}{x}}}\),
so by the Squeeze Theorem,
\(\displaystyle\lim_{x\to+\infty}{\frac{x+\sin x}{x}=1}\).

::: \{\#htm:indeterminate forms\} Indeterminate Forms of Type
\(0\cdot \infty\) and \(\infty -\infty\) :::

\leavevmode\vadjust pre{\hypertarget{htm:definition}{}}%
Definition

\begin{enumerate}
\def\labelenumi{\arabic{enumi}.}
\tightlist
\item
  The \(\displaystyle\lim_{x\to a}{f(x)g(x)}\) is an indeterminate form
  of type \(0\cdot \infty\) if either \[
  \displaystyle\lim_{x\to a}{f(x)=0}\] and
  \(\displaystyle\lim_{x\to a}{g(x)=+\infty}\) or \(-\infty\), or
\end{enumerate}

\[
\displaystyle\lim_{x\to a}{f(x)=+\infty}
\] or \(-\infty\) and \(\displaystyle\lim_{x\to a}{g(x)=0}\).

\begin{enumerate}
\def\labelenumi{\arabic{enumi}.}
\setcounter{enumi}{1}
\tightlist
\item
  The \(\displaystyle\lim_{x\to a}{f(x)g(x)}\) is an indeterminate form
  of type \(\infty - \infty\) if either
\end{enumerate}

\[
\displaystyle\lim_{x\to a}{f(x)=+\infty}
\] and \(\displaystyle\lim_{x\to a}{g(x)=+\infty}\) or \(-\infty\), or

\[
\displaystyle\lim_{x\to a}{f(x)=-\infty}
\] and \(\displaystyle\lim_{x\to a}{g(x)=+\infty}\).

\leavevmode\vadjust pre{\hypertarget{htm:remark}{}}%
Remark L'Hospital's Rule works only for indeterminate forms of type
\(\frac{0}{0}\) and \(\frac{\infty}{\infty}\). Any other indeterminate
form must be expressed equivalently in one of these two forms if we wish
to apply L'Hospital's Rule. For the new indeterminate forms described
above, these conversions can be performed as described below.

\begin{enumerate}
\def\labelenumi{\arabic{enumi}.}
\tightlist
\item
  If \(\displaystyle\lim_{x\to a}{f(x)=0}\) and
  \(\displaystyle\lim_{x\to a}{g(x)=+\infty}\) or \(-\infty\), write
  \(\displaystyle\lim_{x\to a}{f(x)g(x)}\) as:
\end{enumerate}

\begin{enumerate}
\def\labelenumi{(\alph{enumi})}
\tightlist
\item
  \(\displaystyle\lim_{x\to a}{\frac{f(x)}{\frac{1}{g(x)}}}\), which is
  indeterminate of type \(\frac{0}{0}\), or
\item
  \(\displaystyle\lim_{x\to a}{\frac{g(x)}{\frac{1}{f(x)}}}\), which is
  indeterminate of type \(\frac{\infty}{\infty}\) and apply L'Hospital's
  Rule.
\end{enumerate}

\begin{enumerate}
\def\labelenumi{\arabic{enumi}.}
\setcounter{enumi}{1}
\tightlist
\item
  If \(\displaystyle\lim_{x\to a}{f(x)+g(x)}\) is indeterminate of type
  \(\infty - \infty\), rewrite \(f(x)+g(x)\) as a single expression to
  obtain an indeterminate form of type \(\frac{0}{0}\) or
  \(\frac{\infty}{\infty}\) and apply L'Hospital's Rule.
\end{enumerate}

\leavevmode\vadjust pre{\hypertarget{exm:example}{}}%
Example Evaluate the following limits.

\begin{enumerate}
\def\labelenumi{\arabic{enumi}.}
\tightlist
\item
  \(\displaystyle\lim_{x\to 0^+}{\sin^{-1} (2x) \csc x}\)
  \((0\cdot +\infty)\) Solution
\end{enumerate}

\[
\begin{aligned}
\displaystyle\lim_{x\to 0^+}{\sin^{-1} (2x) \csc x} &=\displaystyle\lim_{x\to 0^+}{\frac{\sin^{-1} (2x)}{\sin x}}\\&=\displaystyle\lim_{x\to 0^+}{\frac{1}{\sqrt{1-4x^2}}}\cdot (2) \\&=\frac{2}{1}\\&=2
\end{aligned}
\] 2.
\(\displaystyle\lim_{0\to \frac{\pi}{2}^-}{\tan \theta \ln (\sin \theta)}\)
\(+\infty \cdot 0\) Solution

\[
\begin{aligned}
\displaystyle\lim_{\theta\to \frac{\pi}{2}^-}{\tan \theta \ln (\sin \theta)}&=\displaystyle\lim_{\theta \to \frac{\pi}{2}^-}{\frac {\ln (\sin \theta)}{\cot \theta}}\\&=\displaystyle\lim_{\theta\to \frac{\pi}{2}^-}{\frac{\left(\frac {1}{\sin \theta}\right)\cos \theta}{-\csc^2 \theta}}\\&=\displaystyle\lim_{\theta\to \frac{\pi}{2}^-}{-\sin \theta\cos \theta}\\&=-1(0)\\&=0
\end{aligned}
\] 3.
\(\displaystyle\lim_{x\to 1^+}{\left(\frac{x}{x-1}-\frac{1}{\ln x}\right)}\)
\(\left(\frac{1}{0^+}-\frac{1}{0^+}\right)\) Solution

\[
\begin{aligned}
\displaystyle\lim_{x\to 1^+}{\left(\frac{x}{x-1}-\frac{1}{\ln x}\right)}&=\displaystyle\lim_{x\to 1^+}{\frac{x\ln x -(x-1)}{(x-1)\ln x}}\\&=\displaystyle\lim_{x\to 1^+}{\frac{x\cdot \frac{1}{x}+\ln x-1}{(x-1)\cdot \frac{1}{x}+\ln x}}\\&=\displaystyle\lim_{x\to 1^+}{\frac{\frac{1}{x}}{\frac{1}{x^2}+\frac{1}{x}}}\\&=\frac{1}{2}
\end{aligned}
\]

\#\#\#Indeterminate Forms of Type \(1^\infty\), \(0^\infty\) and
\(\infty ^0\)

Definition Let \(f\) be a nonconstant function. The
\(\displaystyle\lim_{x\to a}{f(x)^{g(x)}}\) is an indeterminate form of
type

\begin{enumerate}
\def\labelenumi{\arabic{enumi}.}
\item
  \(1^\infty\) if \(\displaystyle\lim_{x\to a}{f(x)=1}\) and
  \(\displaystyle\lim_{x\to a}{g(x)=+\infty}\) or \(-\infty\).
\item
  \(0^\infty\) if \(\displaystyle\lim_{x\to a}{f(x)=0}\), through
  positive values, and \(\displaystyle\lim_{x\to a}{g(x)=0}\).
\item
  \(\infty ^0\) if \(\displaystyle\lim_{x\to a}{f(x)=+\infty}\) and
  \(\displaystyle\lim_{x\to a}{g(x)=0}\).
\end{enumerate}

Remark If \(\displaystyle\lim_{x\to a}{f(x)^{g(x)}}\) is indeterminate
of type \(1^\infty\), \(0^\infty\) and \(\infty ^0\), we write

\[
\displaystyle\lim_{x\to a}{f(x)^{g(x)}}=\displaystyle\lim_{x\to a}{e^{g(x) \ln [f(x)]}}
\] and evaluate \(\displaystyle\lim_{x\to a}{g(x) \ln [f(x)]}\) first.
Then, if

\begin{enumerate}
\def\labelenumi{\arabic{enumi}.}
\item
  \(\displaystyle\lim_{x\to a}{g(x) \ln [f(x)]}=L \in R\), then
  \(\displaystyle\lim_{x\to a}{f(x)^{g(x)}}=e^L\).
\item
  \(\displaystyle\lim_{x\to a}{g(x) \ln [f(x)]}=+\infty\), then
  \(\displaystyle\lim_{x\to a}{f(x)^{g(x)}}=+\infty\).
\item
  \(\displaystyle\lim_{x\to a}{g(x) \ln [f(x)]}=-\infty\), then
  \(\displaystyle\lim_{x\to a}{f(x)^{g(x)}}=0\).
\end{enumerate}

::: \{\#exm: example\} Example Evaluate the following limits.

\begin{enumerate}
\def\labelenumi{\arabic{enumi}.}
\tightlist
\item
  \(\displaystyle\lim_{x\to 0^+}{x^{\sin x}}\) \(0^0\) Solution
\end{enumerate}

First, write \(x^{\sin x}=e^{\sin x \ln x}\). Evaluate first
\(\displaystyle\lim_{x\to 0^+}{\sin x \ln x}\).

\[
\begin{aligned}
\displaystyle\lim_{x\to 0^+}{\sin x \ln x}\\&=\displaystyle\lim_{x\to 0^+}{\frac{\ln x}{\csc x}}\\&=\displaystyle\lim_{x\to 0^+}{\frac{\frac{1}{x}}{-\csc x \cot x}}\\&=\displaystyle\lim_{x\to 0^+}{\frac{-\sin^2 x}{x \cos x}}\\&=\displaystyle\lim_{x\to 0^+}{\frac{-2\sin x \cos x}{x(-\sin x)+\cos x}}\\&=0.
\end{aligned}
\] Hence, \(\displaystyle\lim_{x\to 0^+}{x^{\sin x}}=e^0=1\).

\begin{enumerate}
\def\labelenumi{\arabic{enumi}.}
\setcounter{enumi}{1}
\tightlist
\item
  \(\displaystyle\lim_{x\to +\infty}{\left(1-\frac{3}{x}\right)^{2x}}\)
  \(1^\infty\) Solution
\end{enumerate}

\(\left(1-\frac{3}{x}\right)^{2x}=e^{2x\ln \left(1-\frac{3}{x}\right)}\)

\[
\begin{aligned}
\displaystyle\lim_{x\to +\infty}{2x ln \left(1-\frac{3}{2}\right)}\\&=\displaystyle\lim_{x\to +\infty}{\frac{ln \left(1-\frac{3}{x}\right)}{\frac{1}{2x}}}\\&=\displaystyle\lim_{x\to +\infty}{\frac{1}{1-\frac{3}{x}}}\cdot \left(\frac{3}{x^2}\right)\over-\frac{1}{2x^2}\\&=\displaystyle\lim_{x\to +\infty}{\frac{-6}{1-\frac{3}{x}}}\\&=-6.
\end{aligned}
\] Hence,
\$\displaystyle\lim\_\{x\to +\infty\}\{\left(1-\frac{3}{x}\right)\textsuperscript{\{2x\}\}=e}-6.



\end{document}
